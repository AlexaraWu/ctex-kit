% \iffalse meta-comment
% !TEX program  = XeLaTeX
%<*internal>
\iffalse
%</internal>
%<*readme>
jiazhu
======

`jiazhu` is a LaTeX package written to support `jiazhu` (splitted annotation or
inline cutting note, 夹注/双行夹注 in simplified Chinese, 割注/warichū in Japanese)
typesetting for CJK scripts.

You can read the package manual (in Chinese) for more detailed explanations.

Contributing
------------

This package is a part of the [ctex-kit](https://github.com/CTeX-org/ctex-kit) project.

Issues and pull requests are welcome.

Copyright and Licence
---------------------

    Copyright (C) 2018-2020 by Qing Lee <sobenlee@gmail.com>
    Copyright (C)      2020 by Ruixi Zhang <ruixizhang42@gmail.com>
    ----------------------------------------------------------------------

    This work may be distributed and/or modified under the
    conditions of the LaTeX Project Public License, either
    version 1.3c of this license or (at your option) any later
    version. This version of this license is in
       http://www.latex-project.org/lppl/lppl-1-3c.txt
    and the latest version of this license is in
       http://www.latex-project.org/lppl.txt
    and version 1.3 or later is part of all distributions of
    LaTeX version 2005/12/01 or later.

    This work has the LPPL maintenance status `maintained'.

    The Current Maintainer of this work is Qing Lee.

    This package consists of the file  jiazhu.dtx,
                 and the derived files jiazhu.sty,
                                       jiazhu.pdf,
                                       jiazhu.ins,
                                       jiazhu-test.tex, and
                                       README.md (this file).

%</readme>
%<*internal>
\fi
\begingroup
  \def\temp{LaTeX2e}
\expandafter\endgroup\ifx\temp\fmtname\else
\csname fi\endcsname
%</internal>
%<*install>

\input ctxdocstrip %

\preamble

    Copyright (C) 2018-2020 by Qing Lee <sobenlee@gmail.com>
    Copyright (C)      2020 by Ruixi Zhang <ruixizhang42@gmail.com>
--------------------------------------------------------------------------

    This work may be distributed and/or modified under the
    conditions of the LaTeX Project Public License, either
    version 1.3c of this license or (at your option) any later
    version. This version of this license is in
       http://www.latex-project.org/lppl/lppl-1-3c.txt
    and the latest version of this license is in
       http://www.latex-project.org/lppl.txt
    and version 1.3 or later is part of all distributions of
    LaTeX version 2005/12/01 or later.

    This work has the LPPL maintenance status `maintained'.

    The Current Maintainer of this work is Qing Lee.

--------------------------------------------------------------------------

\endpreamble

\postamble

    This package consists of the file  jiazhu.dtx,
                 and the derived files jiazhu.sty,
                                       jiazhu.pdf,
                                       jiazhu.ins,
                                       jiazhu-test.tex, and
                                       README.md (this file).
\endpostamble

\declarepostamble\emptypostamble
\endpostamble

\generate
  {
%</install>
%<*internal>
    \usedir{source/latex/jiazhu}
    \file{jiazhu.ins}       {\from{\jobname.dtx}{install}}
%</internal>
%<*install>
    \usedir{tex/latex/jiazhu}
    \file{jiazhu.sty}       {\from{\jobname.dtx}{package}}
    \usepreamble\emptypreamble
    \usepostamble\emptypostamble
    \usedir{doc/latex/jiazhu}
    \file{jiazhu-test.tex}  {\from{\jobname.dtx}{test}}
    \nopreamble\nopostamble
    \file{README.md}        {\from{\jobname.dtx}{readme}}
  }

\endbatchfile
%</install>
%<*internal>
\fi
%</internal>
%<package>\NeedsTeXFormat{LaTeX2e}
%<package>\RequirePackage{expl3}
%<+package>\GetIdInfo$Id$
%<package>  {Jiazhu/Warichu Support}
%<package>\ProvidesExplPackage{\ExplFileName}
%<package>  {\ExplFileDate}{0}{\ExplFileDescription}
%<*driver>
\documentclass{ctxdoc}
\usepackage{jiazhu}
\ExplSyntaxOn
\sys_if_platform_windows:TF
  { \jiazhuset { ideohtratio = 220/256 } }
  { \jiazhuset { ideohtratio = 0.8 } }
\ExplSyntaxOff
\jiazhuset { opening = 〔 , closing = 〕 }
\newCJKfontfamily\mincho{HaranoAjiMincho}
\SideBySideExampleSet{xrightmargin=.25\linewidth}
\begin{document}
  \DocInput{\jobname.dtx}
  \IndexLayout
  \PrintIndex
\end{document}
%</driver>
% \fi
%
% \CheckSum{977}
% \GetFileId{jiazhu.sty}
%
% \title{\bfseries\pkg{jiazhu} 宏包}
% \author{李清\and 张瑞熹}
% \date{\filedate\qquad\fileversion\thanks{\ctexkitrev{\ExplFileVersion}.}}
% \maketitle
%
% \begin{documentation}
%
% \section{简介}
%
% \pkg{jiazhu} 是一个 \LaTeX 宏包,用于支持
% 夹注\jiazhu{夹注(简体中文)、夾注(繁體中文),指夹在正文中间的注解。
% 夹注一般为两行,故又名双行夹注(简)、雙行夾注(繁)。}^^A
% 或割注\jiazhu[format=\mincho,jzideohtratio=0.88]{^^A
% 割注(日本語),本文中で,複数行に割書きした注釈.}排版。
% 夹注一般只出现在直排(竖排)的文档中,常见于古籍,
% 横排文档则几乎不用、也不适用夹注。
%
% \section{基本用法}
%
% \jiazhuset { opening = {} , closing = {} }
%
% \begin{function}{\jiazhu}
%   \begin{syntax}
%     \cs{jiazhu} \oarg{键值选项} \Arg{夹注内容}
%   \end{syntax}
% \end{function}
%
% \begin{function}{\jiazhuset}
%   \begin{syntax}
%     \cs{jiazhuset} \Arg{键值选项}
%   \end{syntax}
% \end{function}
%
% \begin{function}{format}
%   \begin{syntax}
%     format = \Arg{格式命令}
%   \end{syntax}
%   指定夹注的样式,比如 \meta{行距更改}、\meta{字体切换} 等命令。
%   如果样式为空值,那么夹注与前后正文的字体保持一致。
%   如果夹注两端有括弧,那么指定的样式也会作用在这对括弧上。
%   例如,
%   \begin{SideBySideExample}
%     正文\jiazhu{夹注现在是宋体字}正文\\[5pt]
%     正文\jiazhu[format=\linespread{1.5}]{夹注行距被扩大了}正文\\[5pt]
%     正文\jiazhu[format=\sffamily]{夹注现在是黑体字}正文
%   \end{SideBySideExample}
% \end{function}
%
% \begin{function}{lines}
%   \begin{syntax}
%     lines = \Arg{正整数}
%   \end{syntax}
%   指定夹注的行数,必须是正整数。默认值是~2。例如,
%   \begin{SideBySideExample}
%     正文\jiazhu[lines=3]{夹注现在变成三行}正文
%   \end{SideBySideExample}
% \end{function}
%
% \begin{function}{ideohtratio}
%   \begin{syntax}
%     ideohtratio = \Arg{数字}
%   \end{syntax}
%   \jiazhuset { ideohtratio = 0.5 }^^A
%   指定汉字的字框高与其字号的比值,是实数
%   (也接受 \opt{ideohtratio=8/10} 这种分数写法)。
%   默认值是~0.5。
%   这篇用户手册在导言区通过 \cs{jiazhuset} 声明了 \opt{ideohtratio=0.8}。
%   \begin{center}
%   \footnotesize
%   \def\textbox{\fbox{\rule[-2.4pt]{0pt}{20pt}\rule{20pt}{0pt}}}
%   \setlength{\fboxrule}{0.25pt}
%   \setlength{\fboxsep}{-\fboxrule}
%   \setbox0=\hbox{\textbox}
%   \setbox2=\hbox{\rule[-5pt]{0.5pt}{70pt}}
%   \ht2=0pt
%   \setbox4=\hbox{$\vcenter{}$}
%   \rule[-0.25pt]{80pt}{0.5pt}\kern-70pt
%   \vtop{^^A
%     \offinterlineskip \copy0 \copy0 \copy0
%     \hbox to 20pt{\hss\copy2 \hss}\kern5pt
%     \hbox to 20pt{\hss$\uparrow$\hss}\kern3pt
%     \hbox to 20pt{\hss 直排基线\hss}\kern3pt
%     \hbox to 20pt{\hss 字框“高”50\%\hss}^^A
%   }^^A
%   \copy0 \copy0 \kern15pt
%   \lower\ht4\rlap{$\leftarrow$ 横排基线,字框高 88\%}
%   \end{center}
%   如果你在横排文档中未经设置就直接输入 \cs{jiazhu}\marg{夹注内容},那么
%   你会惊讶\jiazhu{惊讶【形容词】,感到意外、奇怪。}地发现夹注的位置偏下。
%   这是因为 \opt{ideohtratio=0.5} 的默认值是根据直排文档设定的值。
%   在横排文档中,\opt{ideohtratio} 的值视具体的字体而定,
%   一般在 0.8 至 0.88 之间。例如,
%   \begin{SideBySideExample}[xrightmargin=.27\linewidth]
%     正文\jiazhu{默认的参数设定是以直排优先的}正文\\[5pt]
%     正文\jiazhu[ideohtratio=0.8]{调整参数来对齐横排夹注与正文}正文
%   \end{SideBySideExample}
%
%   这里列出几类常见字体对应的 \opt{ideohtratio} 值:
%   \begin{center}
%   \def\hfilll{\hskip 0pt plus 1filll\relax}^^A 慎用
%   \begin{tabular}{cccc}
%   \toprule
%   字体厂商 & 直排 & 横排 & 备注\\
%   \midrule
%   中易  & 0.5 & 0.859375 \hfilll (220/256)  & Windows 系统字体\\
%   华文  & 0.5 & 0.8      \hfilll (800/1000) & macOS 系统字体\\
%   Adobe & 0.5 & 0.88     \hfilll (880/1000) & 思源系列字体\\
%   \bottomrule
%   \end{tabular}
%   \end{center}
%   该值影响夹注的对齐算法。
% \end{function}
%
% \begin{function}{jzideohtratio}
%   \begin{syntax}
%     jzideohtratio = \Arg{数字}
%   \end{syntax}
%   额外指定夹注汉字的字框高与其字号的比值,是实数
%   (也接受 \opt{jzideohtratio=8/10} 这种分数写法)。
%   如果该值被设为~0,则认为夹注字框高的百分比与正文字框高的百分比相等
%   (即无需另设)。
%   如果正文、夹注分别用了来自不同厂商的字体,横排时字框高的百分比就可能不一样,
%   此时可以通过 \opt{jzideohtratio} 另设夹注字体字框高的百分比。^^A
%   \jiazhu{可以理解 \opt{ideohtratio} 对应正文用的字体,
%   而 \opt{jzideohtratio} 则对应夹注(以及夹注括弧)用的字体。
%   两者一般是相等的,所以只需指定 \opt{ideohtratio}。
%   如果两者不相等就需要分别设置。}
%   默认值是~0。
% \end{function}
%
% \begin{function}{ratio}
%   \begin{syntax}
%     ratio = \Arg{数字}
%   \end{syntax}
%   指定夹注字号与正文字号的比值,是实数
%   (也接受 \opt{ratio=1/2} 这种分数写法)。
%   默认值是~2/3。例如,
%   \begin{SideBySideExample}
%     正文\jiazhu[ratio=0.5]{夹注字号是正文字号的 50\%}正文\\
%     正文\jiazhu[ratio=1/2]{夹注字号是正文字号的 50\%}正文
%   \end{SideBySideExample}
% \end{function}
%
% \begin{function}{beforeskip}
%   \begin{syntax}
%     beforeskip = \Arg{弹性间距}
%   \end{syntax}
%   经文多在夹注两端插入少许空白,以此隔开正文与夹注。
%   \opt{beforeskip} 指定前置于夹注的空白。
%   默认值是 \tn{smallskipamount}。例如,
%   \begin{SideBySideExample}
%     正文\jiazhu[beforeskip=0pt plus 1pt]{夹注前端紧贴正文}正文
%   \end{SideBySideExample}
% \end{function}
%
% \begin{function}{afterskip}
%   \begin{syntax}
%     afterskip = \Arg{弹性间距}
%   \end{syntax}
%   指定后置于夹注的空白。
%   默认值是 \tn{smallskipamount}。例如,
%   \begin{SideBySideExample}
%     正文\jiazhu[afterskip=0pt plus 1pt]{夹注后端紧贴正文}正文
%   \end{SideBySideExample}
% \end{function}
%
% \begin{function}{opening}
%   \begin{syntax}
%     opening = \Arg{前置内容}
%   \end{syntax}
%   指定前置于夹注的括弧。默认不添加。例如,
%   \begin{SideBySideExample}[xrightmargin=.3\linewidth]
%     正文\jiazhu[opening=〖]{夹注前置了左空心括号}正文
%   \end{SideBySideExample}
% \end{function}
%
% \begin{function}{closing}
%   \begin{syntax}
%     closing = \Arg{后置内容}
%   \end{syntax}
%   指定后置于夹注的括弧。默认不添加。例如,
%   \begin{SideBySideExample}[xrightmargin=.3\linewidth]
%     正文\jiazhu[closing=】]{夹注后置了右实心括号}正文
%   \end{SideBySideExample}
% \end{function}
%
% \begin{function}{bracketratio}
%   \begin{syntax}
%     bracketratio = \Arg{数字}
%   \end{syntax}
%   指定括弧字号与夹注字号的比值,是实数
%   (也接受 \opt{bracketratio=5/2} 这种分数写法)。
%   默认值是~2。例如,
%   \begin{SideBySideExample}
%     正文\jiazhu[opening=(,bracketratio=1]{括弧字号等于夹注字号}正文
%   \end{SideBySideExample}
% \end{function}
%
% \begin{function}{baselineshift}
%   \begin{syntax}
%     baselineshift = \Arg{尺寸}
%   \end{syntax}
%   \emph{正常情况下,不需要设定此值!}
%
%   如果夹注本身(或者紧随其后的正文)有超常的高度,
%   那么夹注前后的正文可能会错位。例如,
%   \begin{SideBySideExample}
%     \newcommand\ideobaseline{\llap{\rule[-2.1bp]{76.2625pt}{0.2pt}}}
%     正文\jiazhu[lines=4]{特殊效果}正文错位\ideobaseline
%   \end{SideBySideExample}
%
%   让我们设这个“特殊”夹注所处那一行的行距(\tn{baselineskip})为
%   $b$~plus~$y_b$~minus~$z_b$,
%   又设该夹注与其后同一行正文的最大高度(西文基线以上的部分)为~$h$,
%   则一旦 $h>b-\tn{lineskiplimit}$,就会出现正文错位的情况。
%   此时,你可以指定 \opt{baselineshift} 将这个夹注
%   连同后方的正文一起竖直平移(直排时则水平平移)。
%   正值表示向上平移(直排时向右),负值表示向下平移(直排时向左)。
%   设 \tn{lineskip} 为 $l$~plus~$y_l$~minus~$z_l$,
%   当 $h>b-\tn{lineskiplimit}$ 时,
%   可以声明 \opt{baselineshift=\meta{$h-b+l$}}。
%   在上一例中,
%   $h=(0.8-0.5)\times 10.5\,\mathrm{bp}+2\times 7\,\mathrm{bp}$,
%   $b=1.2\times 1.2\times 10.5\,\mathrm{bp}$,
%   $\tn{lineskiplimit}=0\,\mathrm{pt}$,
%   $l=1\,\mathrm{pt}$。
%   因此,
%   \begin{SideBySideExample}
%     \newcommand\ideobaseline{\llap{\rule[-2.1bp]{76.2625pt}{0.2pt}}}
%     正文\jiazhu[lines=4,baselineshift=3.0376125pt]%
%       {特殊效果}正文对齐\ideobaseline
%   \end{SideBySideExample}
% \end{function}
%
% \begin{function}{halign}
%   \begin{syntax}
%     halign = <(justified)|left|right|centered|distributed>
%   \end{syntax}
%   指定夹注内容的横向对齐样式。默认值是 \opt{justified}。
%   其中,\opt{left}、\opt{right}、\opt{centered}
%   分别为“左对齐”、“右对齐”、“居中对齐”。例如,
%   \begin{SideBySideExample}[xrightmargin=.22\linewidth]
%     \jiazhuset{lines=3,ratio=0.5}
%     正文\jiazhu[halign=left]{This is a three-line example}正文\\[5pt]
%     正文\jiazhu[halign=right]{This is a three-line example}正文\\[5pt]
%     正文\jiazhu[halign=centered]{This is a three-line example}正文
%   \end{SideBySideExample}
%
%   另外,\opt{justified}、\opt{distributed} 都是“两端对齐”,它们的区别是:
%   在 \opt{justified} 样式下,末行左对齐;
%   在 \opt{distributed} 样式下,末行仍然两端对齐。
%   例如,
%   \begin{SideBySideExample}
%     正文\jiazhu[halign=justified]{㈠㈡㈢㈣㈤㈥㈦㈧㈨}正文\\
%     正文\jiazhu[halign=distributed]{㈠㈡㈢㈣㈤㈥㈦㈧㈨}正文
%   \end{SideBySideExample}
%
%   \emph{传统排印一般只使用 \opt{justified},不使用
%   \opt{left}、\opt{right}、\opt{centered} 或 \opt{distributed}。}
% \end{function}
%
% \begin{function}{valign}
%   \begin{syntax}
%     valign = <(middle)|bottom|top>
%   \end{syntax}
%   与 \opt{halign} 类似,\opt{valign} 指定夹注内容的纵向对齐样式。
%   默认值是 \opt{middle},即中线对齐。
%   其中,\opt{bottom}、\opt{top}
%   分别为“底线对齐”、“顶线对齐”。例如,
%   \begin{SideBySideExample}
%     \jiazhuset{lines=1,ratio=0.5}
%     正文\jiazhu[valign=bottom]{夹注现在底线对齐}正文\\
%     正文\jiazhu[valign=top]{夹注现在顶线对齐}正文
%   \end{SideBySideExample}
% \end{function}
%
% \begin{function}{shortcut}
%   \begin{syntax}
%     shortcut  = \Arg{字符}
%     shortcut  = \{<开始字符><结束字符>\}
%     shortcut- = \Arg{字符列表}
%   \end{syntax}
% \end{function}
%
% \end{documentation}
%
% \StopEventually{}
%
% \begin{implementation}
%
% \section{代码实现}
%
%    \begin{macrocode}
%<*package>
%<@@=jiazhu>
%    \end{macrocode}
%
%    \begin{macrocode}
\msg_new:nnn { jiazhu } { l3-too-old }
  {
    Support~package~'expl3'~too~old. \\\\
    Please~update~an~up~to~date~version~of~the~bundles\\\\
    'l3kernel'~and~'l3packages'\\\\
    using~your~TeX~package~manager~or~from~CTAN.
  }
\@ifpackagelater { expl3 } { 2019/03/05 } { }
  { \msg_error:nn { jiazhu } { l3-too-old } }
%    \end{macrocode}
%
%    \begin{macrocode}
\cs_if_exist:NF \NewDocumentCommand
  { \RequirePackage { xparse } }
%    \end{macrocode}
%
% \begin{macro}[int]{\jiazhu:nn}
% 主要函数。
%    \begin{macrocode}
\cs_new_protected:Npn \jiazhu:nn #1
  {
    \group_begin:
%    \end{macrocode}
% 水平模式下需要用到 \cs{l_@@_shift_dim},所以必须在这里提前处理键值选项。
%    \begin{macrocode}
      \tl_if_novalue:nF {#1}
        { \keys_set:nn { jiazhu } {#1} }
      \mode_if_vertical:TF
        { \@@_vertical_mode:n }
        {
          \mode_if_horizontal:TF
            {
              \mode_if_inner:TF
                { \@@_inner_mode:n }
                { \@@_horizontal_mode:n }
            }
            { \@@_inner_mode:n }
        }
  }
%    \end{macrocode}
% \end{macro}
%
% \begin{macro}{\@@_vertical_mode:n}
% 竖直模式下不用计算上一段最后一行的宽度,但要注意处理 \tn{parindent}。
%    \begin{macrocode}
\cs_new_protected:Npn \@@_vertical_mode:n
  {
    \mode_leave_vertical:
    \@@_set_line_width:
    \bool_set_true:N \l_@@_full_line_bool
    \bool_set_false:N \l_@@_before_skip_bool
    \dim_set:Nn \l_@@_remaining_width_dim
      {
        \int_compare:nNnTF \tex_lastnodetype:D = \c_one_int
          { \l_@@_line_width_dim - \tex_parindent:D }
          { \l_@@_line_width_dim }
      }
    \@@_boot:n
  }
\bool_new:N \l_@@_full_line_bool
\dim_new:N \l_@@_remaining_width_dim
%    \end{macrocode}
% \end{macro}
%
% \begin{macro}{\@@_horizontal_mode:n}
% 水平模式下需要获取最后一行的宽度,以便将夹注放到段落的后面。
%    \begin{macrocode}
\cs_new_protected:Npn \@@_horizontal_mode:n
  {
    \@@_extract_previous_line_width:
    \@@_set_line_width:
    \dim_set_eq:NN \l_@@_remaining_width_dim \l_@@_line_width_dim
    \int_compare:nNnTF \tex_lastnodetype:D = { 12 }
      {
        \bool_set_false:N \l_@@_full_line_bool
        \bool_set_true:N \l_@@_before_skip_bool
        \dim_sub:Nn \l_@@_remaining_width_dim
          { \g_@@_previous_line_width_dim }
      }
      {
        \bool_set_true:N \l_@@_full_line_bool
        \bool_set_false:N \l_@@_before_skip_bool
      }
    \@@_boot:n
  }
%    \end{macrocode}
% \end{macro}
%
% \begin{macro}{\@@_inner_mode:n}
% 在内部水平或数学模式下,直接将夹注分行输出。
%    \begin{macrocode}
\cs_new_protected:Npn \@@_inner_mode:n
  {
    \bool_set_false:N \l_@@_full_line_bool
    \int_compare:nNnTF \tex_lastnodetype:D = { -1 }
      { \bool_set_false:N \l_@@_before_skip_bool }
      { \bool_set_true:N \l_@@_before_skip_bool }
    \dim_set_eq:NN \l_@@_remaining_width_dim \c_max_dim
    \@@_boot:n
  }
%    \end{macrocode}
% \end{macro}
%
% \begin{macro}{\@@_set_line_width:}
% 当前行的宽度,\tn{hsize} 与 \tn{linewidth} 可能不一致,我们取其较小值。
%    \begin{macrocode}
\cs_new_protected:Npn \@@_set_line_width:
  {
    \dim_set:Nn \l_@@_line_width_dim
      {
          \dim_min:nn { \tex_hsize:D } { \linewidth }
        - \tex_leftskip:D
        - \tex_rightskip:D
      }
  }
\dim_new:N \l_@@_line_width_dim
%    \end{macrocode}
% \end{macro}
%
% \begin{macro}{\@@_extract_previous_line_width:}
% 我们通过在行间数学模式中的 \tn{predisplaysize} 来获取上一行的宽度。
% 目前的 \tn{parshape} 仅考虑 \LaTeX 的列表环境。
%    \begin{macrocode}
\cs_new_protected:Npn \@@_extract_previous_line_width:
  {
    \group_begin:
      \skip_set_eq:NN \tex_parfillskip:D \c_@@_fil_skip
      \c_math_toggle_token \c_math_toggle_token
        \dim_gset_eq:NN \g_@@_previous_line_width_dim \tex_predisplaysize:D
        \int_set_eq:NN \tex_predisplaypenalty:D  \c_@@_nobreak_int
        \int_set_eq:NN \tex_postdisplaypenalty:D \c_@@_nobreak_int
        \skip_set:Nn \tex_abovedisplayskip:D
          {
            \dim_compare:nNnTF
              { \dim_abs:n { \g_@@_previous_line_width_dim } } < \c_max_dim
              { - \l_@@_shift_dim - \tex_baselineskip:D }
              { - \l_@@_shift_dim }
          }
        \skip_set:Nn \tex_belowdisplayskip:D { - \tex_baselineskip:D }
        \skip_set_eq:NN \tex_abovedisplayshortskip:D \tex_abovedisplayskip:D
        \skip_set_eq:NN \tex_belowdisplayshortskip:D \tex_belowdisplayskip:D
      \c_math_toggle_token \c_math_toggle_token
    \group_end:
    \int_set:Nn \tex_prevgraf:D { \tex_prevgraf:D - 3 }
    \dim_compare:nNnTF
      { \dim_abs:n { \g_@@_previous_line_width_dim } } < \c_max_dim
      {
        \dim_gsub:Nn \g_@@_previous_line_width_dim
          {
            \int_compare:nNnTF \tex_parshape:D = \c_one_int
              { \tex_leftskip:D + 2em + \tex_parshapeindent:D \c_one_int }
              { \tex_leftskip:D + 2em }
          }
        \tex_kern:D \g_@@_previous_line_width_dim
      }
      { \dim_gzero:N \g_@@_previous_line_width_dim }
  }
\dim_new:N \g_@@_previous_line_width_dim
%    \end{macrocode}
% \end{macro}
%
% \begin{variable}{\c_@@_break_int, \c_@@_nobreak_int}
% 强制换行/页和禁止换行/页的 penalty 值。
%    \begin{macrocode}
\int_const:Nn \c_@@_break_int   { -10000 }
\int_const:Nn \c_@@_nobreak_int {  10000 }
%    \end{macrocode}
% \end{variable}
%
% \begin{macro}{\@@_boot:n}
% 准备工作。
%    \begin{macrocode}
\cs_new_protected:Npn \@@_boot:n #1
  {
    \@@_tex_parameter:
    \dim_set:Nn \l_@@_outer_unit_dim { \f@size pt }
%    \end{macrocode}
% 初始化设置:夹注的字号设为 $\cs{l_@@_ratio_fp}\times\tn{f@size}$\,pt,
% 默认的行距取相同值(即设置行间距为零)。
%    \begin{macrocode}
    \fp_set:Nn \l_@@_tmp_fp { \l_@@_ratio_fp * \f@size }
    \fontsize
      { \fp_use:N \l_@@_tmp_fp }
      { \fp_use:N \l_@@_tmp_fp }
%    \end{macrocode}
% 更新当前的 \tn{baselinestretch} 为~1。
%    \begin{macrocode}
    \linespread { 1 }
%    \end{macrocode}
% 思源宋体/思源黑体的简体中文版对标点符号的设计有些独特,横排用的全角逗号的
% 底部低于汉字字框的底端,全角引号的顶部又高于字框的顶端,这会导致 \TeX 往
% 行间插入 \tn{lineskip}。我们局部把 \tn{lineskiplimit} 设成最小的
% dimen,相当于局部禁用 \tn{lineskip} 机制,结果是夹注可能会部分重叠
% (横排就不应该有夹注嘛!希望这个默认的设置能够“防止”横排夹注的滥用)。
%    \begin{macrocode}
    \dim_set:Nn \tex_lineskiplimit:D { - \c_max_dim }
%    \end{macrocode}
% 接下来是用户自定义的 \opt{format},并用 \tn{selectfont} 使字体变更生效。
%    \begin{macrocode}
    \tl_use:N \l_@@_format_tl
    \selectfont
    \dim_set:Nn \l_@@_unit_dim { \f@size pt }
    \hbox_set:Nn \l_@@_text_box
      { \tex_ignorespaces:D #1 \tex_unskip:D }
%    \end{macrocode}
% 往夹注首行开头插入支架。
%    \begin{macrocode}
    \@@_make_jzideoht_strut:
    \hbox_set:Nn \l_@@_text_box
      {
        \@@_insert_jzideoht_strut:
        \hbox_unpack:N \l_@@_text_box
      }
    \@@_make_opening_closing_box:
%    \end{macrocode}
% \tn{strutbox} 默认有 $0.3\times\tn{baselineskip}$ 的深度(横排),除
% \XeLaTeX 外的直排模式下,则有 $0.5\times\tn{baselineskip}$ 的“深度”。
% 应该足够分割夹注用了。
%    \begin{macrocode}
    \dim_set:Nn \tex_splitmaxdepth:D { \box_dp:N \strutbox }
    \skip_set:Nn \l_@@_good_break_skip
      { \c_zero_dim plus 0.5\l_@@_outer_unit_dim }
    \skip_set:Nn \l_@@_unit_stretch_skip
      { \c_zero_dim plus \l_@@_unit_dim }
    \@@_set_valign:
    \@@_set_before_skip:
    \@@_processing:
  }
\fp_new:N \l_@@_tmp_fp
\box_new:N \l_@@_text_box
\dim_new:N \l_@@_unit_dim
\dim_new:N \l_@@_outer_unit_dim
\skip_new:N \l_@@_unit_stretch_skip
%    \end{macrocode}
% \end{macro}
%
% \begin{macro}{\@@_tex_parameter:}
% 为避免警告设置的一些 \TeX 参数。
%    \begin{macrocode}
\cs_new_protected:Npn \@@_tex_parameter:
  {
    \@parboxrestore
    \tex_everypar:D { { \box_set_to_last:N \c_zero_int } }
    \dim_zero:N \tex_emergencystretch:D
    \dim_set_eq:NN \tex_hfuzz:D \c_max_dim
    \dim_set_eq:NN \tex_vfuzz:D \c_max_dim
    \int_set_eq:NN \tex_hbadness:D \c_max_int
    \int_set_eq:NN \tex_vbadness:D \c_max_int
    \int_set:Nn \tex_tolerance:D { 1000 }
    \skip_zero:N \tex_splittopskip:D
    \int_zero:N \tex_linepenalty:D
    \int_zero:N \tex_clubpenalty:D
    \int_zero:N \tex_widowpenalty:D
  }
%    \end{macrocode}
% \end{macro}
%
% \begin{macro}{\@@_make_jzideoht_strut:, \@@_insert_jzideoht_strut:}
% 往每个夹注的首行开头插入一个“支架”。
%    \begin{macrocode}
\cs_new_protected:Npn \@@_make_jzideoht_strut:
  {
    \hbox_set:Nn \l_@@_strut_box
      {
        \tex_vrule:D
%    \end{macrocode}
% 支架的高度等于夹注汉字的字框高,
% 这能给 \opt{baselineshift} 提供更可靠的参考。
%    \begin{macrocode}
          height
            \fp_compare:nNnTF \l_@@_jzideoht_fp = \c_zero_fp
              { \fp_use:N \l_@@_ideoht_fp }
              { \fp_use:N \l_@@_jzideoht_fp }
            \l_@@_unit_dim
%    \end{macrocode}
% 在夹注里 \tn{lineskiplimit} 已经最小了,支架的深度不会起任何作用,
% 我们设成零即可。
%    \begin{macrocode}
          depth \c_zero_dim
%    \end{macrocode}
% 支架的宽度当然是零。^^A 也可以改成 \cs{l_@@_unit_dim} 进行视觉 debug
%    \begin{macrocode}
          width \c_zero_dim
        \scan_stop:
    }
  }
\cs_new_protected:Npn \@@_insert_jzideoht_strut:
  { \hbox_unpack:N \l_@@_strut_box }
\box_new:N \l_@@_strut_box
%    \end{macrocode}
% \end{macro}
%
% \begin{macro}{\@@_make_opening_closing_box:}
%    \begin{macrocode}
\cs_new_protected:Npn \@@_make_opening_closing_box:
  {
    \tl_if_empty:NTF \l_@@_opening_tl
      { \box_clear:N \l_@@_opening_box }
      {
        \hbox_set:Nn \l_@@_opening_box
          {
            \fontsize
              { \fp_use:N \l_@@_bracket_ratio_fp \l_@@_unit_dim }
              { \c_zero_skip }
            \selectfont \l_@@_opening_tl
          }
        \dim_sub:Nn \l_@@_remaining_width_dim
          { \box_wd:N \l_@@_opening_box }
      }
    \tl_if_empty:NTF \l_@@_closing_tl
      { \box_clear:N \l_@@_closing_box }
      {
        \hbox_set:Nn \l_@@_closing_box
          {
            \fontsize
              { \fp_use:N \l_@@_bracket_ratio_fp \l_@@_unit_dim }
              { \c_zero_skip }
            \selectfont \l_@@_closing_tl \tex_unskip:D
          }
      }
  }
\cs_new_protected:Npn \@@_put_opening_box:
  {
    \bool_if:NT \l_@@_before_skip_bool
      { \skip_horizontal:N \l_@@_before_skip }
    \box_if_empty:NF \l_@@_opening_box
      { \@@_put_mark_box:N \l_@@_opening_box }
    \cs_set_eq:NN \@@_put_opening_box: \prg_do_nothing:
  }
\cs_new_protected:Npn \@@_put_closing_box:
  {
    \box_if_empty:NF \l_@@_closing_box
      { \@@_put_mark_box:N \l_@@_closing_box }
  }
\box_new:N \l_@@_opening_box
\box_new:N \l_@@_closing_box
%    \end{macrocode}
% \end{macro}
%
% \begin{macro}{\@@_set_valign:}
% \TeX 排放夹注的位置,是以西文基线为准的:
% \begin{center}
% \footnotesize
% \def\textbox{\fbox{\rule[-3.6pt]{0pt}{30pt}\rule{30pt}{0pt}}}
% \def\jiazhubox{\fbox{\rule[-2.16pt]{0pt}{18pt}\rule{18pt}{0pt}}}
% \setlength{\fboxrule}{0.25pt}
% \setlength{\fboxsep}{-\fboxrule}
% \setbox0=\hbox{\textbox\textbox}
% \setbox2=\hbox{\jiazhubox\jiazhubox\jiazhubox}
% \setbox4=\hbox{$\vcenter{}$}
% \rule[-0.25pt]{150pt}{0.5pt}\kern-135pt
% \copy0 \kern-65pt
% \rule[11.15pt]{70pt}{0.5pt}\kern5pt
% \vbox{\baselineskip=24pt \copy2 \copy2 }\kern-59pt
% \rule[18.59pt]{64pt}{0.5pt}\kern10pt
% \lower\ht4\rlap{$\leftarrow$ 西文基线}
% \end{center}
% 设 $r$ 为汉字字框高与字号的比值(例如横排时 $r=0.88$,而直排时 $r=0.5$),
% 其中正文为~$r_{\mathrm{outer}}$、夹注为~$r_{\mathrm{inner}}$;
% 又设 $s_{\mathrm{outer}}$ 与 $s_{\mathrm{inner}}$ 分别为正文字号与
% 夹注字号;记夹注的行距为~$b$(夹注的行间距即为 $b-s_{\mathrm{inner}}$);
% 最后设夹注有 $n$~行。
%
% 为了实现在正文汉字的各处对齐,夹注需要向下平移,平移量各不相同。
% \begin{itemize}
%   \item 底线对齐时,不需要考虑多行情况,平移量计算公式为
%         \[
%            s_{\mathrm{outer}} \cdot (1 - r_{\mathrm{outer}})
%          - s_{\mathrm{inner}} \cdot (1 - r_{\mathrm{inner}}) .
%         \]
%   \item 中线对齐时,平移量的计算公式为
%         \[
%            (r_{\mathrm{inner}} - 0.5) \cdot s_{\mathrm{inner}}
%          + \frac{(n - 1) \cdot b}{2} - (r_{\mathrm{outer}} - 0.5)
%          \cdot s_{\mathrm{outer}} , \quad n \ge 1.
%         \]
%   \item 顶线对齐时,平移量的计算公式为
%         \[
%            r_{\mathrm{inner}} \cdot s_{\mathrm{inner}} + (n - 1) \cdot b
%          - r_{\mathrm{outer}} \cdot s_{\mathrm{outer}} , \quad n \ge 1.
%         \]
% \end{itemize}
%
%    \begin{macrocode}
\cs_new_protected:Npn \@@_set_valign:n #1
  { \cs_set_eq:Nc \@@_set_valign: { @@_set_valign_ #1 : } }
\cs_new_protected:Npn \@@_set_valign_middle:
  {
%    \end{macrocode}
% 中线对齐时,提前展开 $r-0.5$ 的计算结果,避免后续的重复计算。
%    \begin{macrocode}
    \exp_args:Nxx \@@_set_valign_middle_aux:nn
%    \end{macrocode}
% 首先是 $r_{\mathrm{inner}}-0.5$ 的计算结果。
%    \begin{macrocode}
      {
        \fp_compare:nNnTF \l_@@_jzideoht_fp = \c_zero_fp
          { \fp_eval:n { \l_@@_ideoht_fp   - 0.5 } }
          { \fp_eval:n { \l_@@_jzideoht_fp - 0.5 } }
      }
%    \end{macrocode}
% 其次是 $r_{\mathrm{outer}}-0.5$ 的计算结果。
%    \begin{macrocode}
      { \fp_eval:n { \l_@@_ideoht_fp - 0.5 } }
  }
\cs_new_protected:Npn \@@_set_valign_middle_aux:nn #1#2
  {
%    \end{macrocode}
% 对于夹注内容盒子的下移量,我们直接套用公式。
%    \begin{macrocode}
    \dim_set:Nn \l_@@_box_offset_dim
      {
          #1 \l_@@_unit_dim
        + \int_eval:n { \l_@@_lines_int - 1 } \tex_baselineskip:D / 2
        - #2 \l_@@_outer_unit_dim
      }
%    \end{macrocode}
% 对于夹注两端括弧盒子的下移量,我们修改 $s_{\mathrm{inner}}$ 为括弧字号,
% 并取 $n=1$ 即可。
%    \begin{macrocode}
    \dim_set:Nn \l_@@_mark_offset_dim
      {
          \fp_eval:n { #1 * \l_@@_bracket_ratio_fp } \l_@@_unit_dim
        - #2 \l_@@_outer_unit_dim
      }
  }
\cs_new_protected:Npn \@@_set_valign_bottom:
  {
%    \end{macrocode}
% 底线对齐时,则展开 $1-r_{\mathrm{outer}}$ 的计算结果。
%    \begin{macrocode}
    \exp_args:Nxx \@@_set_valign_bottom_aux:nn
      {
        \fp_compare:nNnTF \l_@@_jzideoht_fp = \c_zero_fp
          { \fp_eval:n { 1 -   \l_@@_ideoht_fp } }
          { \fp_eval:n { 1 - \l_@@_jzideoht_fp } }
      }
      { \fp_eval:n { 1 - \l_@@_ideoht_fp } }
  }
\cs_new_protected:Npn \@@_set_valign_bottom_aux:nn #1#2
  {
%    \end{macrocode}
% 夹注内容盒子的下移量
%    \begin{macrocode}
    \dim_set:Nn \l_@@_box_offset_dim
      {
        #2 \l_@@_outer_unit_dim - #1 \l_@@_unit_dim
      }
    \dim_set:Nn \l_@@_mark_offset_dim
      {
          #2 \l_@@_outer_unit_dim
        - \fp_eval:n { #1 * \l_@@_bracket_ratio_fp } \l_@@_unit_dim
      }
  }
%    \end{macrocode}
% 顶线对齐时
%    \begin{macrocode}
\cs_new_protected:Npn \@@_set_valign_top:
  {
    \exp_args:Nxx \@@_set_valign_top_aux:nn
      {
        \fp_compare:nNnTF \l_@@_jzideoht_fp = \c_zero_fp
          { \fp_eval:n { \l_@@_ideoht_fp   } }
          { \fp_eval:n { \l_@@_jzideoht_fp } }
      }
      { \fp_eval:n { \l_@@_ideoht_fp } }
  }
\cs_new_protected:Npn \@@_set_valign_top_aux:nn #1#2
  {
    \dim_set:Nn \l_@@_box_offset_dim
      {
          #1 \l_@@_unit_dim
        + \int_eval:n { \l_@@_lines_int - 1 } \tex_baselineskip:D
        - #2 \l_@@_outer_unit_dim
      }
    \dim_set:Nn \l_@@_mark_offset_dim
      {
          \fp_eval:n { #1 * \l_@@_bracket_ratio_fp } \l_@@_unit_dim
        - #2 \l_@@_outer_unit_dim
      }
  }
\cs_new_eq:NN \@@_set_valign: \@@_set_valign_middle:
\dim_new:N \l_@@_box_offset_dim
\dim_new:N \l_@@_mark_offset_dim
%    \end{macrocode}
% \end{macro}
%
% \begin{macro}{\@@_put_box:N, \@@_put_mark_box:N}
% 输出夹注盒子,并适当下移保证对齐。
%    \begin{macrocode}
\cs_new_protected:Npn \@@_put_box:N #1
  {
    \box_move_down:nn
      { \l_@@_box_offset_dim }
      { \box_use_drop:N #1 }
  }
\cs_new_protected:Npn \@@_put_mark_box:N #1
  {
    \box_move_down:nn
      { \l_@@_mark_offset_dim }
      { \box_use_drop:N #1 }
  }
%    \end{macrocode}
% \end{macro}
%
% \begin{macro}{\@@_set_before_skip:}
%    \begin{macrocode}
\cs_new_protected:Npn \@@_set_before_skip:
  {
    \tl_if_empty:NTF \l_@@_before_skip_tl
      { \bool_set_false:N \l_@@_before_skip_bool }
      {
        \bool_if:NT \l_@@_before_skip_bool
          {
            \skip_set:Nn \l_@@_before_skip
              { \l_@@_before_skip_tl }
            \dim_compare:nNnF \l_@@_remaining_width_dim = \c_max_dim
              {
                \dim_sub:Nn \l_@@_remaining_width_dim
                  {
                      \l_@@_before_skip
                    - \tex_glueshrink:D \l_@@_before_skip
                  }
              }
          }
      }
  }
\bool_new:N \l_@@_before_skip_bool
\skip_new:N \l_@@_before_skip
%    \end{macrocode}
% \end{macro}
%
% \begin{macro}{\@@_set_lines:n}
%    \begin{macrocode}
\cs_new_protected:Npn \@@_set_lines:n #1
  {
    \int_compare:nNnTF {#1} > \c_zero_int
      {
        \int_set:Nn \l_@@_lines_int {#1}
        \@@_make_parshape:n { \l_@@_lines_int }
      }
      { \msg_error:nn { jiazhu } { invalid-number } }
  }
\cs_new_protected:Npn \@@_make_parshape:n #1
  {
    \cs_set_protected:Npx \@@_parshape:
      {
        \tex_parshape:D
          \int_eval:n { #1 + 1 } ~
          \prg_replicate:nn
            {#1} { \c_zero_dim \l_@@_remaining_width_dim }
          \c_zero_dim \c_max_dim \scan_stop:
      }
  }
\int_new:N \l_@@_lines_int
\cs_new_eq:NN \@@_parshape: \prg_do_nothing:
\msg_new:nnn { jiazhu } { invalid-number }
  { Please~specify~a~positive~integer. }
%    \end{macrocode}
% \end{macro}
%
% \begin{macro}{\@@_set_halign:n, \@@_halign:}
%    \begin{macrocode}
\cs_new_protected:Npn \@@_set_halign:n #1
  { \cs_set_eq:Nc \@@_halign: { @@_halign_ #1 : } }
\cs_new_protected:Npn \@@_halign_justified:
  {
    \skip_zero:N \tex_leftskip:D
    \skip_zero:N \tex_rightskip:D
    \skip_set_eq:NN \tex_parfillskip:D \c_@@_fil_skip
  }
\cs_new_protected:Npn \@@_halign_left:
  {
    \skip_zero:N \tex_leftskip:D
    \skip_set_eq:NN \tex_rightskip:D \l_@@_halign_skip
    \skip_set_eq:NN \tex_parfillskip:D \c_@@_fil_skip
  }
\cs_new_protected:Npn \@@_halign_right:
  {
    \skip_set_eq:NN \tex_leftskip:D \l_@@_halign_skip
    \skip_zero:N \tex_rightskip:D
    \skip_zero:N \tex_parfillskip:D
  }
\cs_new_protected:Npn \@@_halign_centered:
  {
    \skip_set_eq:NN \tex_leftskip:D \l_@@_halign_skip
    \skip_set_eq:NN \tex_rightskip:D \l_@@_halign_skip
    \skip_zero:N \tex_parfillskip:D
  }
\cs_new_protected:Npn \@@_halign_distributed:
  {
    \skip_zero:N \tex_leftskip:D
    \skip_zero:N \tex_rightskip:D
    \skip_zero:N \tex_parfillskip:D
    \int_set_eq:NN \tex_tolerance:D \c_max_int
  }
\skip_new:N \l_@@_halign_skip
\cs_new_eq:NN \@@_halign: \@@_halign_justified:
\skip_const:Nn \c_@@_fil_skip { \c_zero_dim plus 1fil }
\skip_set_eq:NN \l_@@_halign_skip \c_@@_fil_skip
%    \end{macrocode}
% \end{macro}
%
% \begin{macro}{\@@_processing:}
%    \begin{macrocode}
\cs_new_protected:Npn \@@_processing:
  {
      \@@_split_lines:
      \@@_put_closing_box:
      \@@_good_break:
      \@@_hskip:N \l_@@_after_skip_tl
    \group_end:
    \tex_ignorespaces:D
  }
%    \end{macrocode}
% \end{macro}
%
% \begin{macro}{\@@_good_break:}
%    \begin{macrocode}
\cs_new_protected:Npn \@@_good_break:
  {
    \skip_horizontal:N \l_@@_good_break_skip
    \tex_penalty:D -100 ~
    \skip_horizontal:n { -\l_@@_good_break_skip }
  }
\skip_new:N \l_@@_good_break_skip
%    \end{macrocode}
% \end{macro}
%
% \begin{macro}{\@@_hskip:N}
%    \begin{macrocode}
\cs_new_protected:Npn \@@_hskip:N #1
  { \tl_if_empty:NF #1 { \skip_horizontal:n {#1} } }
%    \end{macrocode}
% \end{macro}
%
% \begin{macro}{\@@_split_lines:}
%    \begin{macrocode}
\cs_new_protected:Npn \@@_split_lines:
  {
    \dim_set:Nn \l_@@_width_dim
      {
        \dim_max:nn
          { \l_@@_unit_dim }
          { \box_wd:N \l_@@_text_box / \l_@@_lines_int }
      }
    \dim_compare:nNnTF \l_@@_width_dim > \l_@@_remaining_width_dim
      { \@@_split_parshape_lines: }
      { \@@_typeset_remaining: }
  }
\dim_new:N \l_@@_width_dim
%    \end{macrocode}
% \end{macro}
%
% \begin{macro}{\@@_split_parshape_lines:}
%    \begin{macrocode}
\cs_new_protected:Npn \@@_split_parshape_lines:
  {
    \dim_compare:nNnTF \l_@@_remaining_width_dim < \l_@@_unit_dim
      { \@@_split_parshape_lines_auxi: }
      { \@@_split_parshape_lines_auxii: }
  }
\cs_new_protected:Npn \@@_split_parshape_lines_auxi:
  {
    \@@_fill_newline:
    \dim_set:Nn \l_@@_remaining_width_dim
      { \l_@@_line_width_dim - \box_wd:N \l_@@_opening_box }
    \bool_set_true:N \l_@@_full_line_bool
    \@@_split_lines:
  }
\cs_new_protected:Npn \@@_split_parshape_lines_auxii:
  {
    \vbox_set:Nn \l_@@_text_box
      {
        \bool_if:NTF \l_@@_full_line_bool
          { \int_set:Nn \tex_looseness:D { -1 } }
          { \@@_dim_normalize:N \l_@@_remaining_width_dim }
        \@@_parshape:
        \hbox_unpack_drop:N \l_@@_text_box
      }
    \vbox_set_split_to_ht:NNn \l_@@_typeset_box
      \l_@@_text_box { \l_@@_lines_int \tex_baselineskip:D }
    \vbox_set:Nn \l_@@_typeset_box
      { \vbox_unpack:N \l_@@_typeset_box }
    \bool_if:NF \l_@@_full_line_bool
      { \skip_horizontal:N \l_@@_unit_stretch_skip }
    \@@_typeset:
    \box_if_empty:NF \l_@@_text_box
      { \@@_split_remaining_lines: }
  }
\cs_new_protected:Npn \@@_fill_newline:
  {
    \tex_penalty:D \c_@@_nobreak_int
    \tex_hfil:D    \@@_newline:
  }
\cs_new_protected:Npn \@@_newline:
  { \tex_penalty:D \c_@@_break_int }
\box_new:N \l_@@_typeset_box
%    \end{macrocode}
% \end{macro}
%
% \begin{macro}{\@@_split_remaining_lines:}
%    \begin{macrocode}
\cs_new_protected:Npn \@@_split_remaining_lines:
  {
    \@@_newline:
    \dim_set_eq:NN \l_@@_remaining_width_dim \l_@@_line_width_dim
    \bool_set_true:N \l_@@_full_line_bool
    \@@_extract_hbox:Nn \l_@@_text_box
      { \tex_unskip:D \tex_unskip:D \tex_unpenalty:D }
    \@@_split_lines:
  }
%    \end{macrocode}
% \end{macro}
%
% \begin{macro}{\@@_extract_hbox:Nn}
% 从 \tn{vbox} 中取出 \tn{hbox}。
%    \begin{macrocode}
\cs_new_protected:Npn \@@_extract_hbox:Nn #1#2
  {
    \vbox_set:Nn #1
      {
        \vbox_unpack_drop:N #1
        \box_gset_to_last:N \g_@@_last_box
      }
    \hbox_set:Nn #1
      {
%    \end{macrocode}
% 往继续的夹注首行开头插入支架。
%    \begin{macrocode}
        \@@_insert_jzideoht_strut:
        \hbox_unpack_drop:N \g_@@_last_box #2
      }
  }
\box_new:N \g_@@_last_box
%    \end{macrocode}
% \end{macro}
%
% \begin{macro}{\@@_dim_normalize:N}
%    \begin{macrocode}
\cs_new_protected:Npn \@@_dim_normalize:N #1
  {
    \dim_set:Nn #1
      {
        \int_div_truncate:nn {#1} { \l_@@_unit_dim }
        \l_@@_unit_dim
      }
  }
%    \end{macrocode}
% \end{macro}
%
% \begin{macro}{\@@_typeset:, \@@_typeset_remaining:}
%    \begin{macrocode}
\cs_new_protected:Npn \@@_typeset:
  {
    \@@_put_opening_box:
    \@@_put_box:N \l_@@_typeset_box
  }
\cs_new_protected:Npn \@@_typeset_remaining:
  {
    \@@_dim_normalize:N \l_@@_width_dim
    \dim_set:Nn \l_@@_step_dim
      { \l_@@_unit_dim / \l_@@_lines_int }
    \skip_set_eq:NN \l_@@_halign_skip \l_@@_unit_stretch_skip
    \@@_typeset_remaining_auxi:
  }
\cs_new_protected:Npn \@@_typeset_remaining_auxi:
  {
    \vbox_set:Nn \l_@@_typeset_box
      {
        \@@_halign:
        \dim_set_eq:NN \tex_hsize:D \l_@@_width_dim
        \hbox_unpack:N \l_@@_text_box \par
        \int_gset_eq:NN \g_@@_lines_int \tex_prevgraf:D
      }
    \int_compare:nNnTF \g_@@_lines_int > \l_@@_lines_int
      { \@@_typeset_remaining_auxii: }
      { \@@_typeset_remaining_auxiii: }
  }
\cs_new_protected:Npn \@@_typeset_remaining_auxii:
  {
    \dim_add:Nn \l_@@_width_dim { \l_@@_step_dim }
    \@@_typeset_remaining_auxi:
  }
\cs_new_protected:Npn \@@_typeset_remaining_auxiii:
  {
    \@@_extract_max_width:N \l_@@_typeset_box
    \box_set_wd:Nn \l_@@_typeset_box
      { \dim_min:nn { \l_@@_max_dim } { \l_@@_width_dim } }
    \int_compare:nNnF \g_@@_lines_int = \l_@@_lines_int
      {
        \box_set_ht:Nn \l_@@_typeset_box
          {
            \int_eval:n { \l_@@_lines_int - \g_@@_lines_int }
            \tex_baselineskip:D + \box_ht:N \l_@@_typeset_box
          }
      }
    \@@_typeset:
  }
\dim_new:N \l_@@_step_dim
\int_new:N \g_@@_lines_int
%    \end{macrocode}
% \end{macro}
%
% \begin{macro}{\@@_extract_max_width:N}
% 获取盒子中的实际最大行宽。
%    \begin{macrocode}
\cs_new_protected:Npn \@@_extract_max_width:N #1
  {
    \dim_zero:N \l_@@_max_dim
    \box_if_empty:NF #1
      {
        \box_set_eq:NN \l_@@_tmpa_box #1
        \dim_set:Nn \l_@@_width_dim { \box_wd:N #1 }
        \@@_extract_max_width_auxi:
      }
  }
\cs_new_protected:Npn \@@_extract_max_width_auxi:
  {
    \vbox_set_split_to_ht:NNn \l_@@_tmpb_box
      \l_@@_tmpa_box { \tex_baselineskip:D }
    \@@_extract_hbox:Nn \l_@@_tmpb_box
      { \tex_unskip:D \tex_unskip:D \tex_unpenalty:D }
    \dim_set:Nn \l_@@_max_dim
      {
        \dim_max:nn
          { \l_@@_max_dim }
          { \box_wd:N \l_@@_tmpb_box }
      }
    \dim_compare:nNnT
      { \box_wd:N \l_@@_tmpb_box } > \l_@@_width_dim
      { \@@_extract_max_width_auxii: }
    \box_if_empty:NF \l_@@_tmpa_box
      { \@@_extract_max_width_auxi: }
  }
\cs_new_protected:Npn \@@_extract_max_width_auxii:
  {
    \hbox_set_to_wd:Nnn \l_@@_tmpb_box
      { \l_@@_width_dim }
      { \hbox_unpack_drop:N \l_@@_tmpb_box }
%    \end{macrocode}
% \tn{badness} 等于 \num{1000000} 表示盒子宽度溢出了。
%    \begin{macrocode}
    \int_compare:nNnF \tex_badness:D < { 1 000 000 }
      {
        \dim_add:Nn \l_@@_width_dim { \l_@@_step_dim }
        \@@_extract_max_width_auxii:
      }
  }
\dim_new:N \l_@@_max_dim
%    \end{macrocode}
% \end{macro}
%
% \begin{macro}{\@@_add_shortcut:n, \@@_remove_shortcuts:n}
% 设置和删除快捷命令。保险起见,我们只在文档开始时才实际设置快捷命令。
%    \begin{macrocode}
\AtBeginDocument { \@@_activate_shortcut: }
\cs_new_protected:Npn \@@_activate_shortcut:
  {
    \seq_if_empty:NF \l_@@_shortcut_seq
      {
        \seq_map_function:NN
          \l_@@_shortcut_seq \@@_add_shortcut_document:n
      }
    \cs_set_eq:NN \@@_add_shortcut_aux:n \@@_add_shortcut_document:n
    \cs_set_eq:NN \@@_remove_shortcut:n \@@_remove_shortcut_document:n
  }
\cs_new_protected:Npn \@@_add_shortcut:n #1
  {
    \tl_if_blank:nF {#1}
      { \@@_add_shortcut_aux:n {#1} }
  }
\cs_new_protected:Npn \@@_add_shortcut_aux:n
  { \seq_put_right:Nn \l_@@_shortcut_seq }
\cs_new_protected:Npn \@@_remove_shortcuts:n #1
  {
    \tl_map_inline:nn {#1}
      { \exp_args:Nf \@@_remove_shortcut:n { \int_to_arabic:n { `##1 } } }
  }
\cs_new_protected:Npn \@@_add_shortcut_document:n #1
  {
    \tl_if_single_token:nTF {#1}
      { \@@_set_shortcut_group:N #1 }
      { \@@_set_shortcut_delimiter:NNw #1 \q_stop }
  }
\cs_new_protected:Npn \@@_set_shortcut_group:N #1
  { \exp_args:Nf \@@_set_shortcut_group:n { \int_to_arabic:n { `#1 } } }
\cs_new_protected:Npn \@@_set_shortcut_group:n #1
  {
    \@@_save_shortcut:n {#1}
    \char_set_active_eq:nN {#1} \jiazhu
  }
\cs_new_protected:Npn \@@_set_shortcut_delimiter:NNw #1#2#3 \q_stop
  {
    \@@_set_shortcut_delimiter:NNf #1#2
      { \int_to_arabic:n { `#1 } }
  }
\cs_new_protected:Npn \@@_set_shortcut_delimiter:NNn #1#2
  {
    \str_if_eq:nnTF {#1} {#2}
      { \@@_set_shortcut_delimiter:n }
      { \@@_set_shortcut_delimiter:Nn #2 }
  }
\cs_generate_variant:Nn \@@_set_shortcut_delimiter:NNn { NNf }
\cs_new_protected:Npn \@@_set_shortcut_delimiter:n #1
  {
    \@@_active_char:nn
      { \@@_set_shortcut_delimiter:Nn } {#1} {#1}
  }
\cs_new_protected:Npn \@@_set_shortcut_delimiter:Nn #1#2
  {
    \exp_args:Ncc \@@_set_shortcut_delimiter_aux:NNNn
      { @@_shortcut_ #2 :w }
      { @@_shortcut_ #2 _aux:w }
      #1 {#2}
  }
\cs_new_protected:Npn \@@_set_shortcut_delimiter_aux:NNNn #1#2#3#4
  {
    \@@_save_shortcut:n {#4}
    \cs_set_protected:Npn #1
      { \jiazhuoptiongrabber #2 }
    \cs_set_protected:Npn #2 ##1##2 #3
      { \jiazhu:nn { ##1 } { ##2 } }
    \char_set_active_eq:nN {#4} #1
  }
\NewDocumentCommand \jiazhuoptiongrabber { +m +o }
  { #1 {#2} }
\cs_new_protected:Npn \@@_save_shortcut:n #1
  {
    \prop_get:NnNF \l_@@_shortcut_prop {#1} \l_@@_catcode_tl
      {
        \@@_save_active_char:n {#1}
        \prop_put:Nnx \l_@@_shortcut_prop
          {#1} { \char_value_catcode:n {#1} }
        \char_set_catcode_active:n {#1}
      }
  }
\cs_new_protected:Npn \@@_remove_shortcut:n
  { \seq_remove_all:Nn \l_@@_shortcut_seq }
\cs_new_protected:Npn \@@_remove_shortcut_document:n #1
  {
    \prop_pop:NnNT \l_@@_shortcut_prop {#1} \l_@@_catcode_tl
      {
        \char_set_catcode:nn {#1} { \l_@@_catcode_tl }
        \@@_restore_active_char:n {#1}
      }
  }
\cs_new_protected:Npn \@@_save_active_char:n
  { \@@_active_char_aux:Nn \@@_save_active_char:Nn }
\cs_new_protected:Npn \@@_restore_active_char:n
  { \@@_active_char_aux:Nn \@@_restore_active_char:Nn }
\cs_new_protected:Npn \@@_active_char_aux:Nn #1#2
  {
    \group_begin: \exp_args:NNc \group_end:
    #1 { @@_save_active_ #2 :w } {#2}
  }
\cs_new_protected:Npn \@@_save_active_char:Nn #1
  { \@@_active_char:nn { \cs_set_eq:NN #1 } }
\cs_new_protected:Npn \@@_restore_active_char:Nn #1#2
  {
    \char_set_active_eq:nN {#2} #1
    \cs_set_eq:NN #1 \tex_undefined:D
  }
\group_begin:
  \char_set_catcode_active:n { 0 }
  \cs_new_protected:Npn \@@_active_char:nn #1#2
    {
      \group_begin:
        \char_set_lccode:nn { 0 } {#2}
      \tex_lowercase:D { \group_end: #1 ^^@ }
    }
\group_end:
\tl_new:N \l_@@_catcode_tl
\seq_new:N \l_@@_shortcut_seq
\prop_new:N \l_@@_shortcut_prop
%    \end{macrocode}
% \end{macro}
%
% \begin{macro}{format}
% 定义键值选项。
%    \begin{macrocode}
\keys_define:nn { jiazhu }
  {
    format               .tl_set:N = \l_@@_format_tl ,
    beforeskip           .tl_set:N = \l_@@_before_skip_tl ,
    afterskip            .tl_set:N = \l_@@_after_skip_tl ,
    opening              .tl_set:N = \l_@@_opening_tl ,
    closing              .tl_set:N = \l_@@_closing_tl ,
    ideohtratio          .fp_set:N = \l_@@_ideoht_fp ,
    jzideohtratio        .fp_set:N = \l_@@_jzideoht_fp ,
    ratio                .fp_set:N = \l_@@_ratio_fp ,
    bracketratio         .fp_set:N = \l_@@_bracket_ratio_fp ,
    baselineshift       .dim_set:N = \l_@@_shift_dim ,
    lines                  .code:n = \@@_set_lines:n {#1} ,
    shortcut               .code:n = \@@_add_shortcut:n {#1} ,
    shortcut-              .code:n = \@@_remove_shortcuts:n {#1} ,
    halign             .choices:nn =
      { justified , left , right , centered , distributed }
      { \@@_set_halign:n { \l_keys_choice_tl } } ,
    valign             .choices:nn =
      { middle , top , bottom }
      { \@@_set_valign:n { \l_keys_choice_tl } } ,
    lines        .value_required:n = true ,
    halign       .value_required:n = true ,
    valign       .value_required:n = true ,
    shortcut     .value_required:n = true ,
    shortcut-    .value_required:n = true ,
    beforeskip          .initial:n = \smallskipamount ,
    afterskip           .initial:n = \smallskipamount ,
%    \end{macrocode}
% 我们参考 \url{https://www.w3.org/TR/jlreq/#inline_cutting_note}
% 来设定下面这些数值选项的默认值。
%    \begin{macrocode}
    ideohtratio         .initial:n = 0.5 ,
    jzideohtratio       .initial:n = 0 ,
    ratio               .initial:n = 2 / 3 ,
    bracketratio        .initial:n = 2 ,
    lines               .initial:n = 2 ,
    halign              .initial:n = justified ,
    valign              .initial:n = middle
  }
%    \end{macrocode}
% \end{macro}
%
% \begin{macro}{\jiazhu, \jiazhuset}
% 用户命令。
%    \begin{macrocode}
\NewDocumentCommand \jiazhu { +o +m }
  { \jiazhu:nn {#1} {#2} }
\NewDocumentCommand \jiazhuset { }
  { \keys_set:nn { jiazhu } }
%    \end{macrocode}
% \end{macro}
%
% \LuaTeX{} 单独处理。
%
%    \begin{macrocode}
\sys_if_engine_luatex:F
  {
    \box_new:N \l_@@_tmpa_box
    \box_new:N \l_@@_tmpb_box
    \file_input_stop:
  }
%    \end{macrocode}
%
% \begin{macro}{\@@_extract_max_width:N}
%    \begin{macrocode}
\cs_gset_protected:Npn \@@_extract_max_width:N #1
  {
    \box_if_empty:NTF #1
      { \dim_zero:N \l_@@_max_dim }
      {
        \@@_extract_max_width_lua:N #1
        \dim_set_eq:NN \l_@@_width_dim \l_@@_max_dim
      }
  }
\cs_undefine:N \@@_extract_max_width_auxi:
\cs_undefine:N \@@_extract_max_width_auxii:
%    \end{macrocode}
% \end{macro}
%
% \begin{macro}{\@@_extract_max_width_lua:N}
%    \begin{macrocode}
\group_begin:
\char_set_catcode_space:n { 32 }
\lua_now:n
  {
    jiazhu                = jiazhu or { }
    local jiazhu          = jiazhu
    local getbox          = tex.getbox
    local getcount        = tex.getcount
    local setdimen        = tex.setdimen
    local scan_int        = token.scan_int
    local id_hlist        = node.id("hlist")
    local dnode           = node.direct
    local getlist         = dnode.getlist
    local todirect        = dnode.todirect
    local traverse_id     = dnode.traverse_id
    local rangedimensions = dnode.rangedimensions

    function jiazhu.extract_max_width ()
      local box = getbox(scan_int())
      local width = 0
      for hlist in traverse_id(id_hlist, getlist(todirect(box))) do
        local w = rangedimensions(hlist, getlist(hlist))
        if w > width then width = w end
      end
      setdimen("l_@@_max_dim", width)
    end

    local id = luatexbase.new_luafunction("jiazhu")
    local t = lua.get_functions_table()
    t[id] = jiazhu.extract_max_width

    token.set_lua("@@_extract_max_width_lua:N", id, "global", "protected")
  }
\group_end:
%    \end{macrocode}
% \end{macro}
%
%    \begin{macrocode}
%</package>
%    \end{macrocode}
%
% \end{implementation}
%
% \Finale
%
% \endinput
%
% \DisableImplementation
%
% \begin{implementation}
%
% \section{测试文件}
%
%    \begin{macrocode}
%<*test>
%    \end{macrocode}
% 例子来源于 Dian YIN (yindian@ustc) 的 \pkg{gezhu} 宏包。
%    \begin{macrocode}
\documentclass{ctexart}

\usepackage{jiazhu}

\jiazhuset { ideohtratio = 0.8 , opening = 〔 , closing = 〕 , shortcut = | }

\ExplSyntaxOn
  \cs_if_exist_use:NT \xeCJKsetup
    { { AllowBreakBetweenPuncts } }
\ExplSyntaxOff

\begin{document}

\setlength\parskip{30pt}

%%\sloppy

\loop
\noindent\hrulefill\par
\the\hsize\par

世祖光武皇帝讳秀,字文叔,|{测礼“{祖有功而宗有德}”,光武中兴,故庙称世祖。谥法:“能绍前业曰光,克定祸乱曰武。”伏侯古今注曰:“秀之字曰茂。伯、仲、叔、季,兄弟之次。长兄伯升,次仲,故字文叔焉。”}南阳蔡阳人,|{南阳,郡,今邓州县也。蔡阳,县,故城在今随州枣阳县西南。}高祖九世之孙也,出自景帝生长沙定王发。|{长沙,郡,今潭州县也。}发生舂陵节侯买,|{舂陵,乡名,本属零陵泠道县,在今永州唐兴县北,元帝时徙南阳,仍号舂陵,故城在今随州枣阳县东。事具宗室四王传。}买生郁林太守外,|{郁林,郡,今贵州县。前书曰:“郡守,秦官。秩二千石。景帝更名太守。”}外生钜鹿都尉回,|{钜鹿,郡,今邢州县也。前书曰:“都尉,本{郡尉},秦官也。掌佐守,典武职,秩比二千石。景帝更名都尉。”}回生南顿令钦,|{南顿,县,属汝南郡,故城在今陈州项城县西。前书曰:“令、长,皆秦官也。万户以上为令,秩千石至六百石;不满万户为长,秩五百石至三百石。”}钦生光武。光武年九岁而孤,养于叔父良。

身长七尺三寸,美须眉,大口,隆准,日角。|{隆,高也。许负云:“鼻头为准。”郑玄尚书中候注云:“日角谓庭中骨起,状如日。”}性勤于稼穑,|{种曰稼,敛曰穑。}而兄伯升好侠养士,常非笑光武事田业,比之高祖兄仲。|{仲,合阳侯喜也,能为产业。见前书。}王莽天凤中,|{王莽始建国六年改为天凤。}乃之长安,受尚书,略通大义。|{东观记曰:“受尚书于中大夫庐江许子威。资用乏,与同舍生韩子合钱买驴,令从者僦,以给诸公费。”}

莽末,天下连岁灾蝗,寇盗锋起。|{言贼锋锐竞起。字或作“蜂”,谕多也。}地皇三年,|{天凤六年改为地皇。}南阳荒饥,|{《韩诗外传》曰:“一谷不升曰歉,二谷不升曰饥,三谷不升曰馑,四谷不升曰荒,五谷不升曰大侵。”}诸家宾客多为小盗。光武避吏新野,|{新野属南阳郡,今邓州县。《续汉书》曰:“伯升宾客劫人,上避吏于新野邓晨家。”}因卖谷于宛。|{《东观记》曰:“时南阳旱饥,而上田独收。”宛,县,属南阳郡,故城今邓州南阳县也。}宛人李通等以图谶说光武云:“刘氏复起,李氏为辅。”|{图,《河图》也。谶,符命之书。谶,验也。言为王者受命之征验也。《易·坤灵图》曰:“汉之臣李阳也。”}光武初不敢当,然独念兄伯升素结轻客,必举大事,且王莽败亡已兆,天下方乱,遂与定谋,于是乃市兵弩。十月,与李通从弟轶等起于宛,时年二十八。

十一月,有星孛于张。|{《前书音义》曰:“孛星光芒短,蓬然。张,南方宿也。”《续汉志》曰:“张为周地。星孛于张,东南行即翼、轸之分。翼、轸,楚地,是楚地将有兵乱。后一年正月,光武起兵舂陵,攻南阳,斩阜、赐等,杀其士众数万人。光武都洛阳,居周地,除秽布新之象。”}光武遂将宾客还舂陵。时伯升已会众起兵。初,诸家子弟恐惧,皆亡逃自匿,曰“伯升杀我”。及见光武绛衣大冠,|{董巴《舆服志》曰:“大冠者,谓武冠,武官冠之。”《东观记》曰:“上时绛衣大冠,将军服也。”}皆惊曰“谨厚者亦复为之”,乃稍自安。伯升于是招新市、平林兵,|{新市,县,属江夏郡,故城在今郢州富水县东北。平林,地名,在今随州随县东北。}与其帅王凤、陈牧西击长聚。|{《广雅》曰:“聚,居也,音慈谕反。”《前书音义》曰:“小于乡曰聚。”}光武初骑牛,杀新野尉乃得马。|{《前书》曰,尉,秦官,秩四百石至二百石也。}进屠唐子乡,|{《例》曰:“多所诛杀曰屠。”唐子乡有唐子山,在今唐州湖阳县西南。}又杀湖阳尉。|{湖阳属南阳郡,今唐州县也。《东观记》曰:“刘终诈称江夏吏,诱杀之。”}军中分财物不均,众恚恨,欲反攻诸刘。光武敛宗人所得物,悉以与之,众乃悦。进拔棘阳,|{县名,属南阳郡,在棘水之阳,古谢国也,故城在今唐州湖阳县西北。棘音己力反。}与王莽前队大夫甄阜、|{王莽置六队,郡置大夫一人,职如太守。南阳为前队,河内为后队,颍川为左队,弘农为右队,河东为兆队,荥阳为祈队。队音遂。}属正梁丘赐|{王莽每队置属正一人,职如都尉。}战于小长安,|{《续汉书》曰淯阳县有小长安聚,故城在今邓州南阳县南。}汉军大败,还保棘阳。


\ifdim\hsize > 5cm %
  \advance\hsize by -4pt %
\repeat

\end{document}
%    \end{macrocode}
%
%    \begin{macrocode}
%</test>
%    \end{macrocode}
%
% \end{implementation}
%
% \Finale
%
\endinput
